\chapter{Einleitung}
%
Die Frage, wie sich Lernende freiwillig und effektiv mit Bildungsinhalten auseinandersetzen können, ist zentral für die Gestaltung moderner Bildungssysteme. 
Besonders in der Informatik besteht die Herausforderung darin, Studierende erfolgreich an neue Technologien heranzuführen.
\par
Im Rahmen dieses Praxisprojekts im Fachbereich Medieninformatik wird untersucht, wie ein spielerisches Lernsystem Informatikstudierende dabei unterstützen kann, relevante Technologien effizient zu erlernen.
Sowohl die zugrunde liegende Problemstellung als auch der entwickelte Lösungsansatz werden im Detail beschrieben
\par
Zusätzlich werden die Ergebnisse einer projektbegleitenden Umfrage präsentiert, die Einblicke in die Bedürfnisse und Herausforderungen der Zielgruppe bietet.\enlargethispage{\baselineskip}
%

\section*{Hinweis zur geschlechtergerechten Sprache}
Alle Bezeichnungen von Personen werden durch eine Aktion formuliert, die die Person auszeichnet. 
Beispielsweise wird statt Nutzer das Wort Nutzende verwendet, oder statt Student das Wort Studierender. 
Diese Formulierungen soll stets alle Geschlechter umfassen.

\begin{flushright}
Gummersbach, September 2024
\end{flushright}