%%%%%%%%%%%%%%%%%%%%%%%%%%%%%%%%%%%%%%%%%%%%%%%%%%%%%%%%%%%%%%%%%%%%%%%%%%%%%%%
% Begriffe für glossaries definieren
%%%%%%%%%%%%%%%%%%%%%%%%%%%%%%%%%%%%%%%%%%%%%%%%%%%%%%%%%%%%%%%%%%%%%%%%%%%%%%%

%=== Glossar ==================================================================
\newglossaryentry{git}{
	name={Git},
	description={Ein kostenloses Versionsverwaltungssystem, entwickelt von Linus Torvald.}
}

\newglossaryentry{github}{
	name={GitHub},
	description={Eine Entwickler Plattform zum Speicher, Erstellen und Verwalten von Code.}
}

\newglossaryentry{storymode}{
	name={Story Modus},
	description={Ein Modus des Systems, der eine Geschichte erzählt.}
}

\newglossaryentry{sshkey}{
	name={SSH Key},
	description={Kryptografischer Schlüssel, der für die sichere, passwortlose Anmeldung zwischen Computern über das Secure Shell Protokoll (SSH-Protokoll) genutzt wird.}
}

\newglossaryentry{ide}{
	name={IDE},
	description={Integrierte Entwicklungsumgebung. Bietet meist einen Text Editor, sowie fortgeschrittenere Features zur Syntax Überprüfung beim Programmcode schreiben.}
}

\newglossaryentry{commit}{
	name={Commit},
	description={Vorgang in Versionskontrollsystemen wie Git, bei dem Änderungen an Dateien dauerhaft gespeichert werden. Ein Commit enthält eine Nachricht, die die Änderungen beschreibt, und ermöglicht das Verfolgen des Fortschritts und das Wiederherstellen früherer Versionen.}
}

\newglossaryentry{remote}{
	name={Remote},
	description={Verweis auf ein externes Repository in Versionskontrollsystemen wie Git, das meist auf einem Server liegt.}
}

\newglossaryentry{local}{
	name={Local},
	description={Verweis auf ein lokales Repository in Versionskontrollsystemen wie Git, das meist auf dem eigenem Rechner liegt.}
}

\newglossaryentry{repository}{
	name={Repository},
	description={Speicherort in Versionskontrollsystemen wie Git, der alle Dateien, Ordner und den gesamten Versionsverlauf eines Projekts enthält.}
}

\newglossaryentry{branches}{
	name={Branches},
	description={Abzweigungen einer Version in Versionskontrollsystemen. Auf einem Branch kann der Code parallel zum existierendem Hauptcode bearbeitet werden.}
}

\newglossaryentry{mergeconflicts}{
	name={Merge Konflikte},
	description={Konflikte, die beim Zusammenführen von verschiedenen Versionen des Codes passieren können, wenn parallel an der selben aktuellen Version gearbeitet wird.}
}

\newglossaryentry{api}{
	name={API},
	description={Application Programming Interface (API). Eine Schnittstelle, die einen Datenaustausch zwischen verschiedenen Softwareanwendungen ermöglicht.}
}

\newglossaryentry{html}{
	name={HTML},
	description={Hyper Text Markup Language (HTML). Die standardisierte Auszeichnungssprache zur Strukturierung und Darstellung von Inhalten im Web.}
}

%=== Abkürzungen ==============================================================
\newacronym{svm}{SVM}{support vector machine}

%=== Symbole ==================================================================
\newglossaryentry{sym:force}{
	name=\ensuremath{\vec{F}},
	description={Kraft, vektorielle Größe},
	type=symbols,
}
